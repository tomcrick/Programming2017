%%%% Proceedings format for most of ACM conferences (with the exceptions listed below) and all ICPS volumes.
\documentclass[sigconf]{acmart}
\usepackage{paralist}
\usepackage{url}
\usepackage[hyphenbreaks]{breakurl}

\def\UrlBreaks{\do\/\do-}

%%%% As of March 2017, [siggraph] is no longer used. Please use sigconf (above) for SIGGRAPH conferences.

%%%% Proceedings format for SIGPLAN conferences 
% \documentclass[sigplan, anonymous, review]{acmart}

%%%% Proceedings format for SIGCHI conferences
% \documentclass[sigchi, review]{acmart}

%%%% To use the SIGCHI extended abstract template, please visit
% https://www.overleaf.com/read/zzzfqvkmrfzn

\usepackage{booktabs} % For formal tables


% Copyright
%\setcopyright{none}
%\setcopyright{acmcopyright}
\setcopyright{acmlicensed}
%\setcopyright{rightsretained}
%\setcopyright{usgov}
%\setcopyright{usgovmixed}
%\setcopyright{cagov}
%\setcopyright{cagovmixed}

\copyrightyear{2018}
\acmYear{2018}
\setcopyright{acmlicensed}
\acmConference[Koli Calling '18]{18th Koli Calling International Conference on Computing Education Research}{Koli, Finland}
% ACM had "Computing" instead of Computer, please update
%\acmBooktitle{}
%\acmPrice{15.00}
%\acmDOI{10.1145/3159450.3159547}
%\acmISBN{978-1-4503-5103-4/18/02}
% This slight change to the code may also save 1 or 2 lines of space.

% removes the headers from each page per the preparation instructions, as these are not needed and will be updated with the chairs' actual session names during the pagination/indexing process:
\fancyhead{}

\begin{document}
\title{Ioc TBC}
%\titlenote{}
%\subtitle{Extended Abstract}
%\subtitlenote{}

% need other authors?
\author{James H. Davenport}
\affiliation{%
  \institution{University of Bath}
  \streetaddress{}
  \city{Bath} 
  \country{United Kingdom}
}
\email{j.h.davenport@bath.ac.uk}

\author{Rachid Hourizi}
\affiliation{%
  \institution{University of Bath}
  \streetaddress{}
  \city{Bath} 
  \country{United Kingdom}
}
\email{r.hourizi@bath.ac.uk}

\author{Tom Crick}
\affiliation{%
  \institution{Swansea University}
  \streetaddress{}
  \city{Swansea} 
  \country{United Kingdom}
}
\email{thomas.crick@swansea.ac.uk}

 
% The default list of authors is too long for headers}
%\renewcommand{\shortauthors}{Davenport and Crick}


\begin{abstract}
Abstract here...
\end{abstract}

\keywords{TBC}

\maketitle

% From CfP:
% Short papers (up to 5 pages) focus on dissemination and discussion of new ideas in computing education practice or research that merit wider awareness and discussion within the community. They can present preliminary results of new educational innovations, present and discuss novel educational technologies, report work-in-progress research (including promising systems or tools that have not yet been evaluated and/or adopted extensively), or raise issues of significance for the development of the discipline, such as long-term strategic needs for computing education and curricula. All short papers are expected to have an appropriate coverage of literature to support the ideas and arguments that they present. Because it lacks some elements of a research paper, a short paper is evaluated mainly by its anticipated impact on discussions during the conference and possible future contribution to the field of computing education.

\section{Introduction}
Superficially, the employment outlook for computing graduates in the UK looks excellent. \cite[p.~74]{UKCES2015b} states
\begin{quote}
the digital sub-sector will need 518,000 workers for roles in the three highest skilled occupational groups. However, over the last ten years only 164,000 individuals graduated from a first degree in computer science.
\end{quote}
This is profitable for the individual: according to \cite[Figure 4]{BIS2011a}, ``mathematical and computer sciences'' have the second highest earnings return of all subjects (beaten only by ``medicine and dentistry'').
The country profits from this as well: according to \cite[p.~16]{BIS2011a}, this is, per head, the fourth most beneficial to the Exchequer.
\section{So What's the problem}
Despite the headline success in \cite{BIS2011a}, the employment figures are not great, and the earnings data are patchy.
\subsection{Employment}
Quoting \cite{UKCES2015b}, the author of a UK Government-commissioned report \cite{Shadbolt2016a} writes
\begin{quote}
In this context, apparently high rates of unemployment\footnote{11.7\% six months after graduation (the standard UK measure) at the time of \cite{Shadbolt2016a}, compared with a STEM average of 8.4\%. Note, however, that Computing is 20\% of STEM \cite[Table 1]{Wakeham2016a}, so `STEM-less-Computing' has a 7.6\% unemployment rate.} amongst graduates of Computer Sciences and other STEM\footnote{STEM is ``Science, Technology, Engineering, Mathematics'' for \cite{Shadbolt2016a} and this paper.} courses demanded an explanation. 
\end{quote}
A significant explanation is ``There are notable differences in the characteristics of Computer Sciences entrants compared to entrants in other STEM subjects'' \cite[\P2.6]{Shadbolt2016a}: fewer women, but
\begin{description}
\item[50\% more] mature students,
\item[16\% more]Black and Minority Ethnic (BME) and 
\item[40\% more]students from backgrounds where people have traditionally not participated in HE (LPNs).
\end{description}
Mature, BME and LPN students all find getting jobs more difficult.
\par
However, for those students that do find jobs, the data are better, showing \cite[Figure 6]{Shadbolt2016a} fewer students in ``non-graduate jobs'' or low-earning jobs than in STEM as a whole.
\subsection{Earnings}
If we look beyond purely getting jobs to the earnings\footnote{Clearly not the only measure of job quality, or contribution to society, but at least it's measurable, and has been measured in the LEO dataset \cite{DfE2017a}, which tracks individuals through school, university and into the labour market, combining educational, tax and benefits data.} the position (as described in \cite{DfE2018d}, and presented to the public in \cite{BBC2018f}, which also allows the reader to break down the data by university and subject.) is even less clear on a microscopic level, though on a macroscopic level it bears out much of what \cite{Shadbolt2016a} said.

On the macroscopic level, the reader should consider \cite[Table 5]{DfE2018d}. We focus on the `Men' data as presented here, as there are many more than there are women, though the effects are similar. This shows that an OLS (``Ordinary Least Squares'') fit shown that a man reading Computing would earn 3.3\% more than had he read a subject at random. If one corrects for prior attainment, this rises to 10.5\%, and 12.6\% if other factors are taken into account. For reasons explained in \cite[\S4.2]{DfE2018d}, the authors prefer IPRWA (``Inverse Probability Weighted Regression Adjustment''), and this moves the earning difference to 14.7\%. For men, the overall effect of these adjustments is to move Computing from being middle-of-the-pack \cite[Figure 15]{DfE2018d} to fourth best  \cite[Figure 17]{DfE2018d}, and for women is moves to seventh best  \cite[Figure 16]{DfE2018d}.


Overall framing for the paper, etc

\section{UK Policy Context}
Set the scene -- UK digital skills, CS ed reform, digital economy, etc

Cite previous work on programming~\cite{davenport-et-al:latice2016,murphy-et-al:programming2017,simon-et-al:sigcse2018}

Cite previous policy-related work~\cite{crick+sentance:2011,brown-et-al-sigcse2013,brown-et-al-toce2014,crick+moller-wipsce2015}

\section{Institute of Coding}
Formally announced in \cite{DfE2018a}, but foreshadowed in \cite{HMG2015a}.
% From CfP:
% Short papers (up to 5 pages) focus on dissemination and discussion of new ideas in computing education practice or research that merit wider awareness and discussion within the community. They can present preliminary results of new educational innovations...

\section{Anticipated Impact}

\section{Conclusions and Future Work}

\section{Acknowledgements}
Cite funding? 
The first author is grateful to Matt Dickson (Bath) for discussions about \cite{DfE2018d}, but any mistakes are the authors' alone.


\bibliographystyle{ACM-Reference-Format}
\bibliography{kolicalling2018} 

\end{document}
