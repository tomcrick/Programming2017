% Please always include the abstract.
% The abstract MUST be written accorging to the directives stated in 
% http://programming-journal.org/submission/
% Failure to adhere to the abstract directives may result in the paper
% being returned to the authors.

% Each submission must be accompanied by a plain-language abstract of up to 500 words that presents the key points in the paper in a manner understandable by experienced practitioners and researchers in nearby disciplines. The abstract should avoid mathematical symbols whenever possible, and it must address the following:

%     Context: What is the broad context of the work? What is the importance of the general research area?
%     Inquiry: What problem or question does the paper address? How has this problem or question been addressed by others (if at all)?
%     Approach: What did was done that unveiled new knowledge?
%     Knowledge: What new facts were uncovered? If the research was not results oriented, what new capabilities are enabled by the work?
%     Grounding: What argument, feasibility proof, artifacts, or results and evaluation support this work?
%     Importance: Why does this work matter?
%\input abstract.tex
\begin{abstract}
\newline{\emph{Context:}} The question of what programming language should be taught
first has been fiercely debated since computer science teaching
started in higher education. Failure to grasp programming readily almost certainly implies
failure of the course.\\
{\emph{Inquiry:}} What first programming languages \emph{are} being
taught? There have been regular national-scale surveys in Australia and New Zealand,
but not in the UK. The only US survey we have reports on a small subset of universities.\\
{\emph{Approach:}} This paper reports the results of the first survey
of introductory programming courses ($\mathrm{N} = 80$) taught at UK
universities as part of their first year computer science (or similar)
degree programmes, conducted in the first half of 2016.  We report on
student numbers, programming paradigm, programming languages and
environment/tools used, as well as the reasons for choice of such.\\
{\emph{Knowledge:}} The results in this first UK survey indicate a
dominance of Java at a time when universities are still generally
teaching students who are new to programming (and computer science),
despite the fact that Python is perceived, by the same respondents, to
be both easier to teach as well as to learn.\\
{\emph{Grounding:}} Results of this survey are compared with a related
survey conducted since 2010 (as well as earlier surveys from 2001 and
2003) in Australia and New Zealand.\\
{\emph{Importance:}} this survey provides a starting point for
valuable pedagogic baseline data -- providing context for analysis of
the art, science and engineering of programming -- in
the context of substantial computer science curriculum reform in UK
schools, as well as increasing scrutiny of teaching excellence and
graduate employability for UK universities.
\end{abstract} 

