% -*- coding: utf-8; -*-
% vim: set fileencoding=utf-8 :
\documentclass[english,submission]{programming}
%% First parameter: the language is 'english'.
%% Second parameter: use 'submission' for initial submission, remove it for camera-ready (see 4.1)

%\usepackage[backend=biber]{biblatex}
\usepackage{paralist}
\usepackage{url}
\def\UrlBreaks{\do\/\do-}
\usepackage[pdftex,breaklinks]{hyperref}
\usepackage{import}
\usepackage{tikz}

%
% Packages and Commands specific to article (see 3)
%
% These ones  are used in the guide, replace with your own.
% 
\usepackage{multicol}
\lstdefinelanguage[programming]{TeX}[AlLaTeX]{TeX}{%
  deletetexcs={title,author,bibliography},%
  deletekeywords={tabular},
  morekeywords={abstract},%
  moretexcs={chapter},%
  moretexcs=[2]{title,author,subtitle,keywords,maketitle,titlerunning,authorinfo,affiliation,authorrunning,paperdetails,acks,email},
  moretexcs=[3]{addbibresource,printbibliography,bibliography},%
}%
\lstset{%
  language={[programming]TeX},%
  keywordstyle=\firamedium,
  stringstyle=\color{RosyBrown},%
  texcsstyle=*{\color{Purple}\mdseries},%
  texcsstyle=*[2]{\color{Blue1}},%
  texcsstyle=*[3]{\color{ForestGreen}},%
  commentstyle={\color{FireBrick}},%
  escapechar=`,}
\newcommand*{\CTAN}[1]{\href{http://ctan.org/tex-archive/#1}{\nolinkurl{CTAN:#1}}}
%%

\begin{document}

\title{An Analysis of Introductory Programming Courses at UK Universities}
%\titlerunning{Introductory Programming Courses at UK Universities} %optional, in case that the title is too long; the running title should fit into the top page column

\author{Ellen Murphy}
\authorinfo{\email{e.murphy@bath.ac.uk}}
\affiliation{Institute for Mathematical Innovation, University of
  Bath, UK}
\author{Tom Crick}
\authorinfo{\email{tcrick@cardiffmet.ac.uk}}
\affiliation{Department of Computing \& Information Systems, Cardiff
  Metropolitan University, UK}
\author[a]{James H. Davenport}
\authorinfo{\email{j.h.davenport@bath.ac.uk}}
\affiliation{Department of Computer Science, University of
  Bath, UK}

% \authorrunning{E. Murphy, T. Crick, J. H. Davenport} % Optional, for long author lists

\keywords{introductory programming, computing education, higher education, UK} % please provide 1--5 keywords


%%%%%%%%%%%%%%%%%%
%% These data MUST be filled for your submission. (see 4.3)
\paperdetails{
  %% perspective options are: art, sciencetheoretical, scienceempirical, engineering.
  %% Choose exactly the one that best describes this work. (see 2.1)
  perspective=scienceempirical,
  %% State one or more areas, separated by a comma. (see 2.2)
  %% Please see list of areas in http://programming-journal.org/cfp/
  %% The list is open-ended, so use other areas if yours is/are not listed.
  area={Programming education},
}
%%%%%%%%%%%%%%%%%%

%%%%%%%%%%%%%%%%%%
%% These data are provided by the editors. May be left out on submission.
%\paperdetails{
%  submitted=2016-08-10,
%  published=2016-10-11,
%  year=2016,
%  volume=1,
%  issue=1,
%  articlenumber=1,
%}
%%%%%%%%%%%%%%%%%%

%%%%%%%%%%%%%%%%%%%%%%%%%%%%%
% Please go to https://dl.acm.org/ccs/ccs.cfm and generate your Classification
% System [view CCS TeX Code] stanz and copy _all of it_ to this place.
%% From HERE
 \begin{CCSXML}
<ccs2012>
<concept>
<concept_id>10003456.10003457.10003527.10003531</concept_id>
<concept_desc>Social and professional topics~Computing education programs</concept_desc>
<concept_significance>500</concept_significance>
</concept>
<concept>
<concept_id>10003456.10003457.10003527.10003531.10003533</concept_id>
<concept_desc>Social and professional topics~Computer science education</concept_desc>
<concept_significance>500</concept_significance>
</concept>
<concept>
<concept_id>10003456.10003457.10003527.10003531.10003533.10011595</concept_id>
<concept_desc>Social and professional topics~CS1</concept_desc>
<concept_significance>500</concept_significance>
</concept>
<concept>
<concept_id>10003456.10003457.10003527.10003531.10003534</concept_id>
<concept_desc>Social and professional topics~Computer engineering education</concept_desc>
<concept_significance>500</concept_significance>
</concept>
<concept>
<concept_id>10003456.10003457.10003527.10003531.10003751</concept_id>
<concept_desc>Social and professional topics~Software engineering education</concept_desc>
<concept_significance>500</concept_significance>
</concept>
<concept>
<concept_id>10003456.10003457.10003527.10003531.10003537</concept_id>
<concept_desc>Social and professional topics~Computational science and engineering education</concept_desc>
<concept_significance>100</concept_significance>
</concept>
<concept>
<concept_id>10011007.10011006.10011008</concept_id>
<concept_desc>Software and its engineering~General programming languages</concept_desc>
<concept_significance>500</concept_significance>
</concept>
</ccs2012>
\end{CCSXML}

\ccsdesc[500]{Social and professional topics~Computing education programs}
\ccsdesc[500]{Social and professional topics~Computer science education}
\ccsdesc[500]{Social and professional topics~CS1}
\ccsdesc[500]{Social and professional topics~Computer engineering education}
\ccsdesc[500]{Social and professional topics~Software engineering education}
\ccsdesc[100]{Social and professional topics~Computational science and engineering education}
\ccsdesc[500]{Software and its engineering~General programming languages}


% To HERE
%%%%%%%%%%%%%%%%%%%%%%%

\maketitle

% Please always include the abstract.
% The abstract MUST be written accorging to the directives stated in 
% http://programming-journal.org/submission/
% Failure to adhere to the abstract directives may result in the paper
% being returned to the authors.

% Each submission must be accompanied by a plain-language abstract of up to 500 words that presents the key points in the paper in a manner understandable by experienced practitioners and researchers in nearby disciplines. The abstract should avoid mathematical symbols whenever possible, and it must address the following:

%     Context: What is the broad context of the work? What is the importance of the general research area?
%     Inquiry: What problem or question does the paper address? How has this problem or question been addressed by others (if at all)?
%     Approach: What did was done that unveiled new knowledge?
%     Knowledge: What new facts were uncovered? If the research was not results oriented, what new capabilities are enabled by the work?
%     Grounding: What argument, feasibility proof, artifacts, or results and evaluation support this work?
%     Importance: Why does this work matter?

\begin{abstract}
This paper reports the results of the first survey of introductory
programming courses ($\mathrm{N} = 80$) taught at UK universities as
part of their first year computer science (or similar) degree
programmes, conducted in the first half of 2016. Results of this
survey are compared with a related survey conducted since 2010 (as
well as earlier surveys from 2001 and 2003) in Australia and New
Zealand. We report on student numbers, programming paradigm,
programming languages and environment/tools used, as well as the
reasons for choice of such.

The results in this first UK survey indicate a dominance of Java at a
time when universities are still generally teaching students who
are new to programming (and computer science), despite the fact that
Python is perceived to be both easier to teach as well as to
learn. Furthermore, this survey provides a starting point for valuable
pedagogic baseline data in the context of substantial computer science
curriculum reform in UK schools, as well as increasingly scrutiny of
teaching excellence and graduate employability for UK universities.
\end{abstract}


\section{Introduction}\label{intro}

For many years -- and increasingly at all levels of compulsory and
post-compulsory education -- the choice of programming language to
introduce basic programming principles, constructs, syntax and
semantics has been regularly revisited. Even in the context of what
are perceived to be the most difficult introductory topics in computer
science degrees, numerous key themes across programming frequently
appear~\cite{dale:2006}. 

So what is a good first programming language? The issues surrounding
choosing a first language~\cite{gupta:2004,kaplan:2010} -- and a
search of the ACM Digital Library identified a number of papers of the
form ``{\emph{X as a first programming language}}'', going as far back
as the 1970s -- appear to be legion, especially with discussions of
what precisely we aim to achieve from teaching
programming~\cite{fincher:1999,schult+bennedsen:2006}, through to the
potential impact on students' grades and
attainment~\cite{simon-et-al:2006,bergin+reilly:2006,porter-et-al:2013,ivanovic-et-al:2015}. It
appears that decades of research on the teaching of introductory
programming has had limited effect on classroom
practice~\cite{pears-et-al:2007}; although relevant research exists
across several disciplines including education and cognitive science,
disciplinary differences have made this material inaccessible to many
computing educators. Furthermore, computer science instructors have
not had access to comprehensive surveys of research in this
area~\cite{mccracken-et-al:2001,pears-et-al:2007}.

However, in Australia and New Zealand there have been longitudinal
data
collections~\cite{deraadt-et-al:2004,mason-et-al:2012,mason+cooper:2014}
surveying the teaching of introductory programming courses in
universities. Surprisingly, such surveys have not been conducted
elsewhere on this scale, and this paper reports the findings from
running the first such similar survey in the UK.

We remind the reader that the UK consists of four nations with an
overall population of 64.5 million: England (54.3 million), Scotland
(5.3 million), Wales (3.1 million) and Northern Ireland (1.8
million). In 1997, Scotland and Wales held referendums which
determined in both cases the desire for increased self-government
(along with Northern Ireland and the 1998 Good Friday Agreement),
creating assemblies to which a variety of powers -- in particular,
education -- were devolved from the UK Parliament. Thus, we now have
an educational policy ecosystem historically ruled by one parliament
but now consists of three devolved assemblies responsible for four
separate education systems.

In the context of increasing international focus on curriculum and
qualification reform to support computer science education and digital
skills in schools, the four education systems of the UK have proposed
and implemented a variety of
changes~\cite{rs:2012,brown-et-al-sigcse2013,brown-et-al-toce2014},
particular in England, with a new compulsory computing curriculum for
ages 5-16 from September 2014 In the context of increasing
international focus on curriculum and qualification reform to support
computer science education and digital skills in schools, the four
education systems of the UK have proposed and implemented a variety of
changes~\cite{rs:2012,brown-et-al-sigcse2013,brown-et-al-toce2014},
particular in England, with a new compulsory computing curriculum for
ages 5-16 from September 2014~\cite{dfecomp:2013}. For universities
across the UK offering computer science degrees~\cite{qaacomp:2016},
this school curriculum reform has had uncertain (and emerging) impact
on the delivery of their undergraduate programmes, with the diversity
of the educational background of applicants likely to increase before
it narrows: it is certainly possible now for prospective students to
have anywhere from zero to four or five years experience (and
potentially two school qualifications) in computer science with some
knowledge of programming.

Over the past three years, there has been increasing scrutiny of the
quality of teaching in UK universities, partly linked to the current
levels -- and potential future increases -- of tuition fees (generally
paid by the student through government-supported loans), as well as
relative levels of graduate employability and the perceived value of
professional body accreditation by industry. In February 2015, the UK
Department of Business, Innovation \& Skills initiated independent
reviews of STEM degree accreditation and graduate
employability\footnote{\url{https://www.gov.uk/government/collections/graduate-employment-and-accreditation-in-stem-independent-reviews}},
with a specific focus -- the Shadbolt review~\cite{shadbolt:2016} --
on computer science degree accreditation and graduate employability,
reporting back in May 2016. Alongside a number of recommendations to
address the relatively high unemployment rates of computer sciences
graduates, particular on the quality of data, course types, gender and
demographics, the Shadbolt review split generalist universities in
England into three bands, based on their average (across all subjects)
entrance tariff (qualifications of entrants); we have followed that
banding during our analysis the English results, so our data should
allow comparisons.

Thus, in this evolving environment of new policy and curricula, as
well as the emerging demands of innovative pedagogies and high-quality
learning and teaching for computer science degree programmes, we
present the findings from the first national scale survey of
introductory programming languages at UK universities. Through this
first UK wide survey of universities, we identify and analyse trends
in student numbers, programming paradigm, programming languages and
environment/tools used, as well as the reasons for choice of such are
reported.

% Other aspects of first programming courses such as instructor
% experience, 
%external delivery of courses: there was none, so irrelevant
%and resources given to students are also examined, 
% along with comparisons to the  Australasian surveys.

\begin{figure}
\begin{center}
\subimport{plots/}{tariffGroupCompareWide.tex}
\caption{The number of responding universities per Nation/   
 Tariff Group.\label{fig:TG}}
\end{center}
\end{figure}

%\subsection{Student Numbers}

% \begin{table}[]
% \centering
% \caption{The number of programming languages used in first programming courses.\label{tab:numLangs}}
% \label{tab:numLanguages}
% \begin{tabular}{ccccc}
% \hline
% Languages & 1  & 2  & 3 & 4 \\ \hline
% Courses   & 59 & 17 & 3 & 1 \\ \hline
% \end{tabular}
% \end{table}


\section{Methodology}\label{method}

\subsection{Recruitment of Participants}

To recruit for the survey, a general call for participants was sent
out to the Council of Professors and Heads of Computing (CPHC)
membership; CPHC is the representative body for university computer
science departments in the UK, with nearly 800 members at over 100
institutions\footnote{\url{https://cphc.ac.uk/who-we-are/}}. The
survey was hosted online and was available from mid-May until the end
of June 2016; the invitation asked for the survey to be passed on to and completed by
the most appropriate person in that institution. Due
to the recruitment method, there were a number of duplicate responses
from certain departments, and these were reconciled by direct enquiry.

\begin{figure}
\begin{center}
\subimport{plots/}{langPercentCompareWide.tex}
\end{center}
\caption{Language popularity by percentage of courses and students (excl. OU).\label{fig:lang}}
\end{figure}

The questions used in the survey were generously provided by the
authors of the 2013 Australia and New Zealand
survey~\cite{mason+cooper:2014}, so as to allow direct comparison with
the results of this survey. Where possible, questions were left
unchanged, although a small minority were edited to reflect the UK
target audience. As defined in the 2013 Aus/NZ survey, the terminology
``course'' was used for ``{\emph{the basic unit of study that is
completed by students towards a degree, usually studied over a period
of a semester or session, in conjunction with other units of
study}}''.

The first section of the survey asked about the programming
language(s) in use, the reasons for their choice, and their perceived
difficulty and usefulness. Then, questions regarding the use of
environments or development tools; which ones were used, the reasons
for their choice and the perceived difficulty. General questions about
paradigm, instructor experience and external delivery were asked,
along with questions regarding students receiving unauthorised
assistance, and the resources provided to students. Finally,
participants were asked to identify their top three main aims when
teaching introductory programming, and were also allowed to provide
further comments.

% @James: I checked the paper - they could rank all reasons, not just top three.  
In the 2013 Aus/NZ survey, participants were asked to rank
the importance of the reasons for choosing a programming language,
environment or tool. Due to technical limitations in the online survey
tool used, it was not possible to do so in this survey, so
Figure~\ref{fig:reasons} only reports counts. Most questions were not
mandatory; the exceptions were ``{\emph{what programming language(s)
are in use?}}'' and a small number of feeder questions to allow the
survey to function correctly.


\section{Results and Discussion}\label{results}

\subsection{Universities and Courses}
%JHD: we need some discussion of response rates, either here or via
%Figure 1 also showing response rates, as in e-mail to Ellen

Upon completion of the survey, 155 instructors had, at least, started
the survey. Sixty-one of these dropped out before answering the
mandatory questions, and a further 14 were duplicates. Therefore, the
results presented here are drawn from the responses of 80 instructors
from at least 70 institutions. Some participants did not answer all
questions and due to this the response rate varies by question.

\begin{figure}
\begin{center}
\subimport{plots/}{reasonsByCourseCompareWide.tex}
\end{center}
\caption{Reasons given for choosing a programming language by percentage for: all languages; Java; and Python.\label{fig:reasons}}
\end{figure}

Excluding the Open University's 3200 students, the participants in the
survey represented 13462 students, with a mean of 173 (but a standard
deviation of 88). Looking at Figure \ref{fig:TG} we see good response
rates, apart from the specialist higher education institutions (most
of whom do not teach computing) and the ``low tariff'' English
ones. Fewer of these teach computing; this factor alone explains the
response rate. In Northern Ireland, we had responses from the two
universities, but not the university colleges, which are historically
teacher-training colleges.


\subsection{Languages}\label{langs}

%\subsubsection{Choice of Language(s)}

One of the mandatory questions in the survey, and a major point of
interest related to the programming languages in use in introductory
programming courses. Participants were asked to select languages from
a list of 22 programming languages and also had the option to choose
``{\emph{Other}}'' and specify a language not included in the list. The
majority of courses surveyed (59 out of 80) use only one programming
language, with 17 using two (and only three and one institutions using
three and four languages respectively). From the 80 courses, the total
number of {\emph{language instances}} is 106, as some courses use more
than one language to teach introductory programming.

Of the 22 languages provided, 13 were selected at least once. The
relative popularity of languages is shown in Figure~\ref{fig:lang},
where the prevalence is given by the percentage of a language over all
language instances (106 total), and weighted by student numbers (16662
total) per language instance. The programming languages that were not
selected at all were: Actionscript, Ada, Delphi, Eiffel, Fortran,
jBase, Lisp, Ruby and Visual Basic.

\begin{figure}[ht]
\begin{center}
\subimport{plots/}{langByTariffPercentWide.tex}
\end{center}%\vskip-12pt  JHD: Oddly, this vskip seems counterproductive
%\caption{The breakdown of programming languages for each of the Tariff Groups.}
%\end{figure}
%
%\begin{figure}
\begin{center}
\subimport{plots/}{TariffByLangPercentWide.tex}
\end{center}
\caption{The breakdown of programming languages by Nation and Tariff Groups.\label{fig;LangTariff}}
\end{figure}


\section*{Acknowledgements}

The authors would like to thank the participants for their engagement
with the survey, as well as R. Mason and G. Cooper from Southern Cross
University, Australia, for providing us with their survey and
permission to use it. We are grateful to the GW4 Alliance
(Universities of Bath, Bristol, Cardiff and Exeter) for funding the
survey.

\bibliography{programming2017}

\end{document}

% Local Variables:
% TeX-engine: luatex
% End:
